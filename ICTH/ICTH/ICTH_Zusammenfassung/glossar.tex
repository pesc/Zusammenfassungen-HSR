% Glossar and acronyms
\newdualentry{cp}
{CP}
{Control Plane}
{
	Die Control Plane definiert die Logik wie der Verkehr weitergeleitet wird. Sie wird softwaremässig gesteuert (Protokolle, Virtuelle Switches, Virtuelle Router)
}

\newdualentry{dp}
{DP}
{Data Plane / Forwarding Plane}
{
	Die Data Plane leitet die pakete anhand der Logik in der Control Plane weiter. Sie ist die Hardware (Switches, Router)
}

\newdualentry{dte}
{DTE}
{Data Terminal Equipment}
{Client Router}

\newdualentry{dce}
{DCE}
{Data Communication Equipment}
{Modem}

\newdualentry{dslam}
{DSLAM}
{Digital Subscriber Line Access Multiplexer}
{Ist die Vermittlungsstelle zwischen mehrere Endkunden (DSL Anschlüsse) und dem Provider.}

\newdualentry{iana}
{IANA}
{Internet Assigned Numbers Authority}
{Eine Organisation die für die globale Koordination von IP-Adressen zuständig ist. Zudem registriert die IANA viele in Spezifikationen von Netzwerkprotokollen enthaltene Codecs (z.B Ports)}

\newdualentry{rir}
{RIR}
{Regional Internet Registrar}
{Ist für die Verteilung von IP-Adressen innerhalb einer Region zuständig und erhält von der IANA einen definierten Adressrange. Für Europa ist die RIPE der RIR.}

\newdualentry{ripe}
{RIPE}
{Réseaux IP Européens Network Coordination Centre}
{Zuständig für die Vergabe von IP-Adressbereichen und AS-Nummern in Europa}

\newdualentry{icann}
{ICANN}
{Internet Corporation for Assigned Names and Numbers}
{Koordiniert die Vergabe von einmaligen Namen und Adressen im Internet (DNS)}

% Glossar entries
\newglossaryentry{gls-cos}
{
	name={verbindungsorientierter Dienst},
	description={Bei einem verbindungsorientierten Dienst wird im vorhinein eine Verbindung aufgebaut (z.B TCP, Frame Relay) Dabei werden immer die drei Phasen Verbindungsaufbau, Datenübertragung und Verbindungsabbau durchlaufen. }
}

\newglossaryentry{gls-cls}
{
	name={verbindungsloser Dienst},
	description={Ein verbindungsloser Dienst startet dagegen direkt mit der Übertragung von Daten. (z.B UDP, Ethernet)}
}