\section{Architekturthemen}

\subsection{Dependency Injection}
\begin{itemize}
    \item View-Models in einem X-Plattform-Projekt haben keinen Zugriff auf UI-Komponenten und können somit:
    \begin{itemize}
        \item keine Fenster öffnen
        \item keine UI Events direkt behandeln
        \item generell keine platformspezifischen Dinge erledigen
    \end{itemize}
    \item Um das dennoch zu machen $\Rightarrow$ Dependency Injection
\end{itemize}

Wie erreichen wir den Zugriff auf plattformspezifischen Code?
\begin{enumerate}
    \item Wir abstrahieren den Zugriff mittels Interface
\begin{lstlisting}
public interface INavigationService 
{ 
  void Show(string view, object vm); 
} 
\end{lstlisting}
    \item Wir programmieren im ViewModel-Code gegen diese Interfaces
\begin{lstlisting}
public class AutoListVm : BindableBase {
  // Service in Property speichern
  public INavigationService NavigationService { get; }
  //Service im Konstruktor übergeben
  public AutoListVm(INavigationService navService) { 
    // Service der Propery zuweisen
    NavigationService = navService;
  } 
  public void ShowDetailView() {
    // Service verwenden
    NavigationService.Show("Auto", SelectedAuto); 
} } 
\end{lstlisting}
    \item Wir implementieren im Plattform-Projekt für jedes abstrakte Interface einen konkreten Service
\begin{lstlisting}
// Service Klasse
public class WpfNavigationService : INavigationService 
{ 
  public readonly Dictionary<string, Action<object>> WindowFactory = new Dictionary<string, Action<object>>(); 
  
  public WpfNavigationService() 
  { 
    WindowFactory.Add("AutoList", vm => AutoListWindow.Display(vm as AutoListVm)); 
    WindowFactory.Add("Auto", vm => AutoWindow.Display(vm as AutoVm)); 
    WindowFactory.Add("Customer", vm => CustomerWindow.Display(vm as CustomerVm)); 
  } 
  public void Show(string view, object vm) 
  { 
    var factoryAction = WindowFactory[view];
    factoryAction(vm); 
  } 
} 
\end{lstlisting}
    \item Wir instanziieren die verwendeten Service im Startup-Code des Plattform-Projekts
    \item Wir injizieren den ViewModels die beötigten Services (z.B via Konstruktor-Parameter)
\begin{lstlisting}
public partial class App : Application 
{ 
  public AppVm Vm { get; private set; } 
  public INavigationService NavigationService { get; private set; } 
  protected override void OnStartup(StartupEventArgs e) 
  { 
    base.OnStartup(e); 
    NavigationService = new WpfNavigationService(); 
    Vm = new AppVm(NavigationService); 
    ... 
  }
} 
\end{lstlisting}
\end{enumerate}
\paragraph{Verbesserungmöglichkeiten}
Anstatt Magic-String wie etwa ''AutoList'' zu verwendet, können folgende bessere Wege verwendet werden:
\begin{enumerate}
    \item String-Konstanten 
    \item Enums
    \item Typ-Parameter (Generics)
\end{enumerate}
Die dritte Variante ist empfohlen:

\begin{lstlisting}
public interface INavigationService { void Show<T>(T vm); }

public class WpfNavigationService : INavigationService { 
  public readonly Dictionary<Type, Action<object>> WindowFactory = new Dictionary<Type, Action<object>>();
  public WpfNavigationService() { 
    WindowFactory.Add(typeof(AutoListVm), vm => AutoListWindow.Display(vm as AutoListVm)); WindowFactory.Add(typeof(AutoVm), vm => AutoWindow.Display(vm as AutoVm)); 
    WindowFactory.Add(typeof(CustomerVm), vm => CustomerWindow.Display(vm as CustomerVm)); } 
    public void Show<T>(T vm) {
      var factoryAction = WindowFactory[typeof(T)];
      factoryAction(vm);
} } 
\end{lstlisting}


\paragraph{Vorteile / Nachteile}
\begin{itemize}
    \item[+]  Zwang zur ''Separation of Concerns'' 
    \item[+] Konfigurierbare Lebensdauer von Objekten 
    \item[+] Konfigurierbare Multiplizität von Objektinstanzen (z.B. Singleton, Object-Pool) 
    \item[+] Flexiblere Testbarkeit (Mock-Services) 
    \item[+] Einfacherer Austausch einer Service-Implementierung durch eine andere 
    \item[-] Konfigurationsfehler fallen erst zur Laufzeit auf 
    \item[-] (Scheinbar) höhere Komplexität 
    \item[-] Mehr Code 
    \item[-] Code wird (un)übersichtlicher
    \item[-] Dadurch Code auch schwieriger zu debuggen
\end{itemize}


\subsection{Background Execution}
Standardmässig laufen 2 Threads:
\begin{description}
    \item[UI-Thread] Event Handling, Layout Management
    \item[Render-Thread] Zeichnen der Controls auf den Screen
\end{description}
\begin{itemize}
    \item Fenster werden neu gezeichnet auch während einer ''Blockade''
\end{itemize}

\paragraph{Tasks}
\begin{lstlisting}
Task.Run(() =>
{
    // Anwendung im Background
    // UI-Thread läuft parallel weiter
});
\end{lstlisting}

\paragraph{Dispatcher}
\\
Synchron
\begin{lstlisting}
Dispatcher.Invoke(() =>
{
    // UI blockiert
});
\end{lstlisting}
Asynchron
\begin{lstlisting}
Dispatcher.BeginInvoke(() =>
{
    // UI nicht blockiert
});
\end{lstlisting}


\subsection{Background Operation}
\begin{enumerate}
    \item Visuelles Feedback geben (Spinner, Overlay, Popup)
    \item Starten eines BG-Threads, Kontrolle geht gleich an UI Thread zurück
    \item Bei Thread-Ende aus dem BG-Thread den UI-Thread irgendwie benachrichtigen
\end{enumerate}
\begin{lstlisting}
public void CreateAsciiArtinBackgroundThread()
{
    Task.Run(() =>
    {
        // Berechnung läuft nun in einem Background Thread
        CreateAsciiArt();
        // am Schluss den Dispatcher nutzen, um eine abschliessende Aktion im UI-Thread auszuführen: 
        Dispatcher.Invoke(() => { ... });
    });
}
\end{lstlisting}




