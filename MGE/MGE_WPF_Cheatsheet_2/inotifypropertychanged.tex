\textbf{INotifyPropertyChanged}
\begin{lstlisting}[language=java]
// Clock.cs
public class Clock:INotifyPropertyChanged
{
  private DateTime _time;
  public DateTime Time {
    get { return _time; }
    set {
      if (value != _time) {
        _time = value;
        OnPropertyChanged(nameof(Time));
        OnPropertyChanged(nameof(TimeString));
      } 
    }
  }
  public string TimeString => Time.ToString("dd.MM.yyyy HH:mm:ss");
  public Clock() {
    Time = DateTime.Now;
  }
  public event PropertyChangedEventHandler PropertyChanged;
  protected virtual void OnPropertyChanged(string propertyName = null) {
    PropertyChanged?.Invoke(this, new PropertyChangedEventArgs(propertyName));
  }
}

//App.xaml.cs
public partial class App : Application
{
  public DispatcherTimer Timer { get; set; }
  public Clock Clock { get; set; }
  protected override void OnStartup(StartupEventArgs e) {
    Clock = new Clock();
    Timer = new DispatcherTimer();
    // Jede Sekunde das Tick-Event auslösen:
    Timer.Interval = TimeSpan.FromSeconds(1);
    Timer.Tick += (sender, args) => {
      // Do something...
      Clock.Time = DateTime.Now;
    };
    Timer.Start();
    MainWindow = new MainWindow();
    MainWindow.DataContext = Clock;
    MainWindow.Show();
  }
}
\end{lstlisting}

\begin{lstlisting}[language=xml]
<!-- MainWindow.xaml -->
<Label ... Content="{Binding TimeString}" />
\end{lstlisting}