\documentclass[
a4paper,
oneside,
10pt,
fleqn,
headsepline,
toc=listofnumbered, 
bibliography=totocnumbered]{scrartcl}

% deutsche Trennmuster etc.
\usepackage[T1]{fontenc}
\usepackage[utf8]{inputenc}
\usepackage[english, ngerman]{babel} % \selectlanguage{english} if  needed
\usepackage{lmodern} % use modern latin fonts

% Custom commands
\newcommand{\GITHUB}{https://github.com/jklaiber/HSR}
\newcommand{\LICENSEURL}{https://en.wikipedia.org/wiki/Beerware}
\newcommand{\LICENSE}{
"THE BEER-WARE LICENSE" (Revision 42):
Julian Klaiber and Severin Dellsperger wrote this file. As long as you retain this notice you
can do whatever you want with this stuff. If we meet some day, and you think
this stuff is worth it, you can buy us a beer in return.
}

% Jede Überschrift 1 auf neuer Seite
\let\stdsection\section
\renewcommand\section{\clearpage\stdsection}

% Multiple Authors
\usepackage{authblk}

% Include external pdf
\usepackage{pdfpages}

% Layout / Seitenränder
\usepackage{geometry}

% Inhaltsverzeichnis
\usepackage{makeidx} 
\makeindex

\usepackage{url}
\usepackage[pdfborder={0 0 0}]{hyperref}
\usepackage[all]{hypcap}
\usepackage{hyperxmp} % for license metadata

% Glossar und Abkürzungsverzeichnis
\usepackage[acronym,toc,nopostdot]{glossaries}
\setglossarystyle{altlist}
\usepackage{xparse}
\DeclareDocumentCommand{\newdualentry}{ O{} O{} m m m m } {
	\newglossaryentry{gls-#3}{
		name={#4 : #5},
		text={#5\glsadd{#3}},
		description={#6},
		#1
	}
	\makeglossaries
	\newacronym[see={[Siehe:]{gls-#3}},#2]{#3}{#4}{#5\glsadd{gls-#3}}
}
\makeglossaries

% Mathematik
\usepackage{amsmath}
\usepackage{amssymb}
\usepackage{amsfonts}
\usepackage{enumitem}

% Images
\usepackage{graphicx}
\graphicspath{{images/}} % default paths

% Boxes
\usepackage{fancybox}

%Tables
\usepackage{tabu}
\usepackage{booktabs} % toprule, midrule, bottomrule
\usepackage{array} % for matrix tables
\usepackage{multicol} %multicol

% Header and footer
\usepackage{scrlayer-scrpage}
\setkomafont{pagehead}{\normalfont}
\setkomafont{pagefoot}{\normalfont}
\automark*{section}
\clearpairofpagestyles
\ihead{\headmark}
\ohead{\AUTHOR}
\cfoot{\pagemark}

% Pseudocode
\usepackage{algorithmic}
\usepackage[linesnumbered,ruled]{algorithm2e}

% Code Listings
\usepackage{listings}
\usepackage{color}
\usepackage{beramono}

\definecolor{bluekeywords}{rgb}{0,0,1}
\definecolor{greencomments}{rgb}{0,0.5,0}
\definecolor{redstrings}{rgb}{0.64,0.08,0.08}
\definecolor{xmlcomments}{rgb}{0.5,0.5,0.5}
\definecolor{types}{rgb}{0.17,0.57,0.68}

\lstdefinestyle{visual-studio-style}{
	language=[Sharp]C,
	columns=flexible,
	showstringspaces=false,
	basicstyle=\footnotesize\ttfamily, 
	commentstyle=\color{greencomments},
	morekeywords={partial, var, value, get, set},
	keywordstyle=\bfseries\color{bluekeywords},
	stringstyle=\color{redstrings},
	breaklines=true,
	breakatwhitespace=true,
	tabsize=4,
	numbers=left,
	numberstyle=\tiny\color{black},
	frame=lines,
	showspaces=false,
	showtabs=false,
	escapeinside={£}{£},
}

\definecolor{DarkPurple}{rgb}{0.4, 0.1, 0.4}
\definecolor{DarkCyan}{rgb}{0.0, 0.5, 0.4}
\definecolor{LightLime}{rgb}{0.3, 0.5, 0.4}
\definecolor{Blue}{rgb}{0.0, 0.0, 1.0}

\lstdefinestyle{cevelop-style}{
	language=C++,  
	columns=flexible,
	showstringspaces=false,     
	basicstyle=\footnotesize\ttfamily, 
	keywordstyle=\bfseries\color{DarkPurple},
	commentstyle=\color{LightLime},
	stringstyle=\color{Blue}, 
	escapeinside={£}{£}, % latex scope within code      
	breaklines=true,
	breakatwhitespace=true,
	showspaces=false,
	showtabs=false,
	tabsize=4,
	morekeywords={include,ifndef,define},
	numbers=left,
	numberstyle=\tiny\color{black},
	frame=lines,
}

\lstdefinestyle{eclipse-style}{
	language=Java,  
	columns=flexible,
	showstringspaces=false,     
	basicstyle=\footnotesize\ttfamily, 
	keywordstyle=\bfseries\color{DarkPurple},
	commentstyle=\color{LightLime},
	stringstyle=\color{Blue}, 
	escapeinside={£}{£}, % latex scope within code      
	breaklines=true,
	breakatwhitespace=true,
	showspaces=false,
	showtabs=false,
	tabsize=4,
	morekeywords={length},
	numbers=left,
	numberstyle=\tiny\color{black},
	frame=lines,
}
\lstset{style=eclipse-style}



% Theorems \begin{mytheo}{title}{label}
\usepackage{tcolorbox}
\tcbuselibrary{theorems}
\newtcbtheorem[number within=section]{definiton}{Definition}%
{fonttitle=\bfseries}{def}
\newtcbtheorem[number within=section]{remember}{Merke}%
{fonttitle=\bfseries}{rem}
\newtcbtheorem[number within=section]{hint}{Hinweis}%
{fonttitle=\bfseries}{hnt}

% Colors
\definecolor{strings}{HTML}{448c25}
\definecolor{comments}{HTML}{aaaaaa}
\definecolor{keywords}{HTML}{aa3d8c}
\definecolor{background}{HTML}{f4f4f4}
\definecolor{numbers}{HTML}{a884e0}

% Default style
\lstdefinestyle{default}{
    backgroundcolor=\color{background},
    basicstyle=\ttfamily\small,
    breakatwhitespace=true,
    breaklines=true,
    commentstyle=\color{comments},
    deletekeywords={},
    escapeinside={}{},
    extendedchars=true,
    frame=lines,
    keepspaces=true,
    keywordstyle=\color{keywords},
    morekeywords={},
    numbers=left,
    numberstyle=\ttfamily\color{numbers},
    rulecolor=\color{numbers},
    showspaces=false,
    showstringspaces=false,
    showtabs=false,
    stepnumber=1,
    stringstyle=\color{strings},
    tabsize=2,
}
\lstset{
    style=default,
    columns=fullflexible
}

% Language cisco-config
\lstdefinelanguage{cisco-config}{
    morekeywords={no,ip,ipv6,int,interface},
    morecomment=[l][\color{comments}]{!},
    numbers=none
}

% Language cisco-teminal
\lstdefinelanguage{cisco-terminal}{
    morecomment=[l][\color{strings}]{\#},
    morecomment=[l][\color{strings}]{>},
    numbers=none
}

\lstdefinelanguage{bash}{
    numbers=none
}

\makeatletter
\@addtoreset{section}{part}
\makeatother

% Boxes
\tcbuselibrary{most}

\usepackage{fontawesome}

% \cmd{...}
\newcommand{\cmd}[1]{\texttt{#1}}

% Info Box
\definecolor{infobar}{HTML}{02cefc}
\definecolor{infobackground}{HTML}{baf0fc}
\newcommand{\info}[2]{
    \begin{tcolorbox}[
        arc = 0mm,
        boxrule = 0pt,
        breakable,
        before skip=11pt,
        before skip=11pt,
        title = \faInfo~#1,
        fonttitle = \sffamily\bfseries,
        coltitle = white,
        colbacktitle = infobar,
        colback = infobackground,
        toptitle=2mm,
        bottomtitle=2mm,
        top=4mm,
        bottom=4mm
    ]
    #2
    \end{tcolorbox}
}

% Warning Box
\definecolor{warnbar}{HTML}{f90053}
\definecolor{warnbackground}{HTML}{fcc4d7}
\newcommand{\warn}[2]{
    \begin{tcolorbox}[
        arc = 0mm,
        boxrule = 0pt,
        breakable,
        before skip=11pt,
        before skip=11pt,
        title = \faWarning~#1,
        fonttitle = \sffamily\bfseries,
        coltitle = white,
        colbacktitle = warnbar,
        colback = warnbackground,
        toptitle=2mm,
        bottomtitle=2mm,
        top=4mm,
        bottom=4mm
    ]
    #2
    \end{tcolorbox}
}

% Login Information Box
\definecolor{loginbar}{HTML}{FA8A05}
\definecolor{loginbackground}{HTML}{F3D4AF}
\newcommand{\login}[2]{
    \begin{tcolorbox}[
        arc = 0mm,
        boxrule = 0pt,
        breakable,
        before skip=11pt,
        before skip=11pt,
        title = \faKey~#1,
        fonttitle = \sffamily\bfseries,
        coltitle = white,
        colbacktitle = loginbar,
        colback = loginbackground,
        toptitle=2mm,
        bottomtitle=2mm,
        top=4mm,
        bottom=4mm
    ]
    #2
    \end{tcolorbox}
}

\definecolor{settingborder}{HTML}{0066CC}
\definecolor{settingcontent}{HTML}{E5F2FA}
\newcommand{\setting}[1]{
    \begin{tcolorbox}[
        colback = settingcontent,
        colframe = settingborder
    ]
    You can find the settings under: \\
    \textbf{\clicks{~#1}}
    \end{tcolorbox}
}

\definecolor{configurationborder}{HTML}{005700}
\definecolor{configurationcontent}{HTML}{e6ffcc}
\newcommand{\configuration}[2]{
    \begin{tcolorbox}[
        colback = configurationcontent,
        colframe = configurationborder
    ]
    \textbf{Configuration:}\\
    To reach an output like below you only have to add/change the following parameters 
    ~#1
    You can find the settings under: \\
    \textbf{\clicks{~#2}}
    \end{tcolorbox}
}


\usepackage{multirow}

% URLs
\urlstyle{rm}
\definecolor{link}{HTML}{0450f2}
\hypersetup{
    colorlinks,
    allcolors=.,
    urlcolor=link,
}

% \url[display]{url} or \url{url}
\renewcommand{\url}[2][]{%
    \ifstrempty{#1}{%
        \burlalt{#2}{#2}%
    }{%
        \burlalt{#1}{#2}%
    }%
}

% Clicks
\newcommand{\clicks}[1]{%
    $\foreach \n [count=\ni] in {#1}{%
        \ifnum\ni=1%
            \textit{\n}%
        \else%
            \rightarrow \textit{\n}%
        \fi%
    }$%
}

% Keystrokes
\newcommand{\keys}[1]{%
    $\foreach \n [count=\ni] in {#1}{%
        \ifnum\ni=1%
            \textit{\n}%
        \else%
            + \textit{\n}%
        \fi%
    }$%
}